% Options for packages loaded elsewhere
\PassOptionsToPackage{unicode}{hyperref}
\PassOptionsToPackage{hyphens}{url}
%
\documentclass[
]{article}
\title{Časovne vrste - seminar}
\usepackage{etoolbox}
\makeatletter
\providecommand{\subtitle}[1]{% add subtitle to \maketitle
  \apptocmd{\@title}{\par {\large #1 \par}}{}{}
}
\makeatother
\subtitle{Poročilo o analizi časovnih vrst}
\author{Brina Pirc in Anja Trobec}
\date{Maj 2022}

\usepackage{amsmath,amssymb}
\usepackage{lmodern}
\usepackage{iftex}
\ifPDFTeX
  \usepackage[T1]{fontenc}
  \usepackage[utf8]{inputenc}
  \usepackage{textcomp} % provide euro and other symbols
\else % if luatex or xetex
  \usepackage{unicode-math}
  \defaultfontfeatures{Scale=MatchLowercase}
  \defaultfontfeatures[\rmfamily]{Ligatures=TeX,Scale=1}
\fi
% Use upquote if available, for straight quotes in verbatim environments
\IfFileExists{upquote.sty}{\usepackage{upquote}}{}
\IfFileExists{microtype.sty}{% use microtype if available
  \usepackage[]{microtype}
  \UseMicrotypeSet[protrusion]{basicmath} % disable protrusion for tt fonts
}{}
\makeatletter
\@ifundefined{KOMAClassName}{% if non-KOMA class
  \IfFileExists{parskip.sty}{%
    \usepackage{parskip}
  }{% else
    \setlength{\parindent}{0pt}
    \setlength{\parskip}{6pt plus 2pt minus 1pt}}
}{% if KOMA class
  \KOMAoptions{parskip=half}}
\makeatother
\usepackage{xcolor}
\IfFileExists{xurl.sty}{\usepackage{xurl}}{} % add URL line breaks if available
\IfFileExists{bookmark.sty}{\usepackage{bookmark}}{\usepackage{hyperref}}
\hypersetup{
  pdftitle={Časovne vrste - seminar},
  pdfauthor={Brina Pirc in Anja Trobec},
  hidelinks,
  pdfcreator={LaTeX via pandoc}}
\urlstyle{same} % disable monospaced font for URLs
\usepackage[margin=1in]{geometry}
\usepackage{graphicx}
\makeatletter
\def\maxwidth{\ifdim\Gin@nat@width>\linewidth\linewidth\else\Gin@nat@width\fi}
\def\maxheight{\ifdim\Gin@nat@height>\textheight\textheight\else\Gin@nat@height\fi}
\makeatother
% Scale images if necessary, so that they will not overflow the page
% margins by default, and it is still possible to overwrite the defaults
% using explicit options in \includegraphics[width, height, ...]{}
\setkeys{Gin}{width=\maxwidth,height=\maxheight,keepaspectratio}
% Set default figure placement to htbp
\makeatletter
\def\fps@figure{htbp}
\makeatother
\setlength{\emergencystretch}{3em} % prevent overfull lines
\providecommand{\tightlist}{%
  \setlength{\itemsep}{0pt}\setlength{\parskip}{0pt}}
\setcounter{secnumdepth}{-\maxdimen} % remove section numbering
\ifLuaTeX
  \usepackage{selnolig}  % disable illegal ligatures
\fi

\begin{document}
\maketitle

DATOTEKA A

\begin{enumerate}
\def\labelenumi{\arabic{enumi}.}
\tightlist
\item
  Narišite graf in komentirajte, ali se iz njega vidi kakšen trend ali
  sezonskost.

  \includegraphics{porocilo_seminar_D_files/figure-latex/unnamed-chunk-2-1.pdf}
  \includegraphics{porocilo_seminar_D_files/figure-latex/unnamed-chunk-2-2.pdf}
\end{enumerate}

\begin{verbatim}
## [1] "Ne opazimo trenda, opazimo pa sezonskost."
\end{verbatim}

\begin{enumerate}
\def\labelenumi{\arabic{enumi}.}
\setcounter{enumi}{1}
\tightlist
\item
  Odstranite morebiten trend in sezonskost z metodami, uporabljenimi pri
  tečaju: (zaporedno) diferenciranje, logaritmiranje, neposredna ocena
  sezonskih komponent, polinomski trend stopnje največ 3 ali prileganje
  periodične funkcije (ali kakšna kombinacija teh metod). Potem ko
  odstranite morebiten trend, narišite tudi surovi in zglajeni
  periodogram ter komentirajte, ali se vidi kakšna sezonskost in kakšna
  naj bi bila perioda.
\end{enumerate}

REŠEVANJE: logaritmiranja ne moremo uporabiti, ker imamo negativne
podatke diferenciranje:

\begin{verbatim}
## 
## Call:
## lm(formula = x ~ I(sin(p * t)) + I(cos(p * t)) + I(sin(p2 * t)) + 
##     I(cos(p2 * t)))
## 
## Residuals:
##     Min      1Q  Median      3Q     Max 
## -583.92 -146.58   -7.66  145.46  545.64 
## 
## Coefficients:
##                Estimate Std. Error t value Pr(>|t|)    
## (Intercept)    770.8550     7.6106 101.287   <2e-16 ***
## I(sin(p * t))  287.0049    10.7703  26.648   <2e-16 ***
## I(cos(p * t))  550.2589    10.7703  51.090   <2e-16 ***
## I(sin(p2 * t))  18.6738    10.7656   1.735   0.0832 .  
## I(cos(p2 * t))  -0.4413    10.7751  -0.041   0.9673    
## ---
## Signif. codes:  0 '***' 0.001 '**' 0.01 '*' 0.05 '.' 0.1 ' ' 1
## 
## Residual standard error: 219.1 on 824 degrees of freedom
## Multiple R-squared:  0.8012, Adjusted R-squared:  0.8002 
## F-statistic: 830.1 on 4 and 824 DF,  p-value: < 2.2e-16
\end{verbatim}

\includegraphics{porocilo_seminar_D_files/figure-latex/unnamed-chunk-3-1.pdf}

\begin{verbatim}
## 
## Call:
## lm(formula = d ~ I(sin(perioda * t)) + I(cos(perioda * t)))
## 
## Residuals:
##      Min       1Q   Median       3Q      Max 
## -204.725  -50.142   -0.614   49.743  226.055 
## 
## Coefficients:
##                     Estimate Std. Error t value Pr(>|t|)    
## (Intercept)          -0.1355     2.5258  -0.054    0.957    
## I(sin(perioda * t)) 450.9594     3.5720 126.247   <2e-16 ***
## I(cos(perioda * t)) 862.4393     3.5720 241.442   <2e-16 ***
## ---
## Signif. codes:  0 '***' 0.001 '**' 0.01 '*' 0.05 '.' 0.1 ' ' 1
## 
## Residual standard error: 72.68 on 825 degrees of freedom
## Multiple R-squared:  0.989,  Adjusted R-squared:  0.989 
## F-statistic: 3.712e+04 on 2 and 825 DF,  p-value: < 2.2e-16
\end{verbatim}

\includegraphics{porocilo_seminar_D_files/figure-latex/unnamed-chunk-3-2.pdf}
\includegraphics{porocilo_seminar_D_files/figure-latex/unnamed-chunk-3-3.pdf}

\includegraphics{porocilo_seminar_D_files/figure-latex/unnamed-chunk-4-1.pdf}
\includegraphics{porocilo_seminar_D_files/figure-latex/unnamed-chunk-4-2.pdf}

\begin{verbatim}
## Daniell(2) 
## coef[-2] = 0.2
## coef[-1] = 0.2
## coef[ 0] = 0.2
## coef[ 1] = 0.2
## coef[ 2] = 0.2
\end{verbatim}

\begin{verbatim}
## mDaniell(1) 
## coef[-1] = 0.25
## coef[ 0] = 0.50
## coef[ 1] = 0.25
\end{verbatim}

\begin{verbatim}
## Daniell(1,1) 
## coef[-2] = 0.1111
## coef[-1] = 0.2222
## coef[ 0] = 0.3333
## coef[ 1] = 0.2222
## coef[ 2] = 0.1111
\end{verbatim}

\begin{verbatim}
## mDaniell(1,1) 
## coef[-2] = 0.0625
## coef[-1] = 0.2500
## coef[ 0] = 0.3750
## coef[ 1] = 0.2500
## coef[ 2] = 0.0625
\end{verbatim}

\includegraphics{porocilo_seminar_D_files/figure-latex/unnamed-chunk-4-3.pdf}
\includegraphics{porocilo_seminar_D_files/figure-latex/unnamed-chunk-4-4.pdf}
\includegraphics{porocilo_seminar_D_files/figure-latex/unnamed-chunk-4-5.pdf}

\begin{verbatim}
##   [1] 216 217 215 218 214 219 213 220 212 221 211 222 210 223 209 224 208 225
##  [19] 207 206 226 205 227 204 228 203 229 202 230 201 231 200 232 199 233 198
##  [37] 234 235 197 236 196 248 250 249 252 251 247 237 253 246 195 254 184 185
##  [55] 183 182 255 181 245 186 256 180 238 179 244 187 194 257 178 177 258 188
##  [73] 398 305 399 397 304 176 385 400 306 307 243 384 386 308 303 401 396 259
##  [91] 387 239 302 321 175 260 193 388 299 322 277 301 298 189 420 300 320 279
## [109] 278 276 174 309 297 395 419 421 383 261 242 402 156 275 389 280 155 319
## [127] 173 323 157 418 262 154 394 296 363 422 240 417 390 310 153 190 274 158
## [145] 393 382 416 324 364 263 281 172 241 403 423 392 295 318 362 152 159 192
## [163] 391 264 273 191 294 171 381 415 361 365 424 160 311 282 151 374 272 380
## [181] 325 293 375 265 317 376 360 404 170 266 271 414 366 373 132 169 425 150
## [199] 379 283 133 292 161 377 267 270 378 333 168 316 372 359 312 131 134 268
## [217] 269 334 413 284 313 367 291 354 351 326 405 353 162 332 355 352 135 149
## [235] 412 285 289 290 371 130 426 358 167 288 315 335 314 368 350 286 163 287
## [253] 165 336 166 406 331 327 357 369 356 411 330 136 129 164 370 337 328 329
## [271] 148 427 349 407 338 137 410 342 345 348 339 346 347 408 343 409 428 344
## [289] 341 340 147 128 138 429 431 430 432 139 146 127 140 108 109 107 145 110
## [307] 106 141 111 105 144 112 142 113 143 104 114 126 103  85 115  84  82  83
## [325]  86  81  87  88 102  89  80  90 116 125  91 101  92  79 117  64 124  65
## [343]  93  63 100  62  66  94  61 118  78  95  67  99 123  96  60  97  98 119
## [361]  68  12  36  13  35  11  77  59  37  34  10  14  33  38  58  69  15   9
## [379] 120  39 122  32  76  57  16  70   8 121  56  40  31  71  17   7  75  41
## [397]  55  72  30  18  54  42   6  74  73  29  53  19  43   5  52  28  44  20
## [415]   4  51  45  27  21  50  46   3  49  22  47  48  26  25  23  24   2   1
\end{verbatim}

\begin{verbatim}
## [1] 0.2500000 0.2511574 0.2488426 0.2523148
\end{verbatim}

\begin{verbatim}
## [1] 4.000000 3.981567 4.018605 3.963303
\end{verbatim}

\includegraphics{porocilo_seminar_D_files/figure-latex/unnamed-chunk-5-1.pdf}

\begin{enumerate}
\def\labelenumi{\arabic{enumi}.}
\setcounter{enumi}{2}
\tightlist
\item
  Narišite graf rezidualov in komentirajte, ali so videti stacionarni.
  Stacionarnost tudi preizkusite z uporabo ustreznih statističnih metod.
\end{enumerate}

\begin{verbatim}
## [1] 0.9889818
\end{verbatim}

\includegraphics{porocilo_seminar_D_files/figure-latex/unnamed-chunk-6-1.pdf}

\begin{verbatim}
## 
##  Augmented Dickey-Fuller Test
## 
## data:  d.res
## Dickey-Fuller = -16.866, Lag order = 9, p-value = 0.01
## alternative hypothesis: stationary
\end{verbatim}

\begin{verbatim}
## [1] "Augmented Dickey-Fuller Test ne zavrne stacionarnosti."
\end{verbatim}

\begin{enumerate}
\def\labelenumi{\arabic{enumi}.}
\setcounter{enumi}{3}
\tightlist
\item
  Na rezidualih naredite grafikona ACF in PACF in na njuni podlagi
  predlagajte vsaj en model vrste AR(p) ali MA(q).
\end{enumerate}

\includegraphics{porocilo_seminar_D_files/figure-latex/unnamed-chunk-8-1.pdf}
\includegraphics{porocilo_seminar_D_files/figure-latex/unnamed-chunk-8-2.pdf}

\begin{verbatim}
## [1] "Predlagava izbiro modela MA(1) in AR(10), torej ARMA(1,10)."
\end{verbatim}

\begin{enumerate}
\def\labelenumi{\arabic{enumi}.}
\setcounter{enumi}{4}
\tightlist
\item
  Na podlagi Yule--Walkerjevih cenilk in kriterija AIC izberite
  najboljši model AR(p). Primerjajte ga z najboljšim modelom ARMA(p, q)
  za p + q ≤ 3 po kriteriju AIC (pozor: kriterij AIC je lahko definiran
  drugače od postopka do postopka). Če je videti smiselno, pa namesto
  tega uporabite model GARCH.
\end{enumerate}

\begin{verbatim}
## 
## Call:
## ar(x = d.res, aic = TRUE, arg = "yule–walker")
## 
## Coefficients:
##       1        2        3        4        5        6        7        8  
## -0.9220  -0.9094  -0.8658  -0.8001  -0.7618  -0.6768  -0.6231  -0.5965  
##       9       10       11       12       13       14       15       16  
## -0.5208  -0.4867  -0.3520  -0.3175  -0.2951  -0.2842  -0.3573  -0.3849  
##      17       18       19       20       21       22       23       24  
## -0.3418  -0.3625  -0.3654  -0.3199  -0.3217  -0.3150  -0.2899  -0.2264  
##      25       26  
## -0.1609  -0.0685  
## 
## Order selected 26  sigma^2 estimated as  2852
\end{verbatim}

\begin{verbatim}
## 
## Call:
## arima(x = d.res, order = c(0, 0, 1))
## 
## Coefficients:
##           ma1  intercept
##       -0.9897     0.0537
## s.e.   0.0060     0.0214
## 
## sigma^2 estimated as 2748:  log likelihood = -4455.07,  aic = 8916.14
\end{verbatim}

\begin{verbatim}
## [1] "Izbrali sva model MA(1)."
\end{verbatim}

\begin{enumerate}
\def\labelenumi{\arabic{enumi}.}
\setcounter{enumi}{5}
\tightlist
\item
  Izberite »optimalni« model in ocenite vse njegove parametre.
\end{enumerate}

\begin{verbatim}
## [1] "OPTIMALNI MODEL: izberemo tisti model, ki ima najnižji aic. V najinem primeru je to MA(1)."
\end{verbatim}

\includegraphics{porocilo_seminar_D_files/figure-latex/unnamed-chunk-10-1.pdf}
\includegraphics{porocilo_seminar_D_files/figure-latex/unnamed-chunk-10-2.pdf}

\begin{verbatim}
## 
##  Shapiro-Wilk normality test
## 
## data:  best$residuals
## W = 0.99795, p-value = 0.4167
\end{verbatim}

\includegraphics{porocilo_seminar_D_files/figure-latex/unnamed-chunk-10-3.pdf}

\begin{verbatim}
## [1] "Shapirov test ne zavrne hipoteze, torej gre za normalno porazdelitev kar je očitno tudi iz grafa."
\end{verbatim}

\begin{enumerate}
\def\labelenumi{\arabic{enumi}.}
\setcounter{enumi}{6}
\tightlist
\item
  Oglejte si ostanke po vašem modelu in komentirajte, ali so videti kot
  beli šum. Primerjajte njihovo porazdelitev z normalno.

  \includegraphics{porocilo_seminar_D_files/figure-latex/unnamed-chunk-11-1.pdf}
\end{enumerate}

\begin{verbatim}
## 
##  Box-Pierce test
## 
## data:  d.res
## X-squared = 179.27, df = 1, p-value < 2.2e-16
\end{verbatim}

\begin{verbatim}
## 
##  Box-Ljung test
## 
## data:  d.res
## X-squared = 179.92, df = 1, p-value < 2.2e-16
\end{verbatim}

\begin{verbatim}
## [1] "Ne gre za white noise."
\end{verbatim}

\begin{enumerate}
\def\labelenumi{\arabic{enumi}.}
\setcounter{enumi}{7}
\tightlist
\item
  Z uporabo izbranega modela in pod predpostavko normalnosti z R-ovo
  funkcijo predict konstruirajte 90\% napovedni interval za naslednjo
  vrednost. Ne pozabite vračunati tudi odstranjenega trenda in
  sezonskosti.
\end{enumerate}

\includegraphics{porocilo_seminar_D_files/figure-latex/unnamed-chunk-12-1.pdf}
\includegraphics{porocilo_seminar_D_files/figure-latex/unnamed-chunk-12-2.pdf}
\includegraphics{porocilo_seminar_D_files/figure-latex/unnamed-chunk-12-3.pdf}

\begin{enumerate}
\def\labelenumi{\arabic{enumi}.}
\setcounter{enumi}{8}
\tightlist
\item
  Dobljeni napovedni interval primerjajte z napovednim intervalom, ki bi
  ga dobili, če bi naivno privzeli, da so podatki kar Gaussov beli šum
  -- pred in po odstranitvi trenda in sezonskosti.
\end{enumerate}

\includegraphics{porocilo_seminar_D_files/figure-latex/unnamed-chunk-13-1.pdf}
\includegraphics{porocilo_seminar_D_files/figure-latex/unnamed-chunk-13-2.pdf}
\includegraphics{porocilo_seminar_D_files/figure-latex/unnamed-chunk-13-3.pdf}
\includegraphics{porocilo_seminar_D_files/figure-latex/unnamed-chunk-13-4.pdf}

DATOTEKA B, če logaritmiramo

\begin{enumerate}
\def\labelenumi{\arabic{enumi}.}
\tightlist
\item
  \emph{Narišite graf in komentirajte, ali se iz njega vidi kakšen trend
  ali sezonskost.}

  \includegraphics{porocilo_seminar_D_files/figure-latex/unnamed-chunk-14-1.pdf}
\end{enumerate}

\begin{verbatim}
## [1] "Opazimo trend, na prvi pogled ne opazimo sezonskosti."
\end{verbatim}

\begin{enumerate}
\def\labelenumi{\arabic{enumi}.}
\setcounter{enumi}{1}
\tightlist
\item
  \emph{Odstranite morebiten trend in sezonskost z metodami,
  uporabljenimi pri tečaju: (zaporedno) diferenciranje, logaritmiranje,
  neposredna ocena sezonskih komponent, polinomski trend stopnje največ
  3 ali prileganje periodične funkcije (ali kakšna kombinacija teh
  metod). Potem ko odstranite morebiten trend, narišite tudi surovi in
  zglajeni periodogram ter komentirajte, ali se vidi kakšna sezonskost
  in kakšna naj bi bila perioda.}
\end{enumerate}

\includegraphics{porocilo_seminar_D_files/figure-latex/unnamed-chunk-15-1.pdf}

\begin{verbatim}
## 
## Call:
## lm(formula = d ~ t)
## 
## Residuals:
##     Min      1Q  Median      3Q     Max 
## -5.7489 -0.2438  0.0435  0.3718  1.1512 
## 
## Coefficients:
##             Estimate Std. Error t value Pr(>|t|)    
## (Intercept) 3.598206   0.083546  43.069  < 2e-16 ***
## t           0.004720   0.000555   8.506 1.48e-15 ***
## ---
## Signif. codes:  0 '***' 0.001 '**' 0.01 '*' 0.05 '.' 0.1 ' ' 1
## 
## Residual standard error: 0.6716 on 258 degrees of freedom
## Multiple R-squared:  0.219,  Adjusted R-squared:  0.216 
## F-statistic: 72.35 on 1 and 258 DF,  p-value: 1.481e-15
\end{verbatim}

\begin{verbatim}
## 
## Call:
## lm(formula = d ~ t + I(cos(perioda2 * t)) + I(cos(perioda3 * 
##     t)) + I(cos(perioda4 * t)) + I(sin(perioda5 * t)) + I(cos(perioda6 * 
##     t)))
## 
## Residuals:
##     Min      1Q  Median      3Q     Max 
## -5.0183 -0.1792  0.0701  0.3551  0.9637 
## 
## Coefficients:
##                        Estimate Std. Error t value Pr(>|t|)    
## (Intercept)           3.5969896  0.0775436  46.387  < 2e-16 ***
## t                     0.0047298  0.0005151   9.182  < 2e-16 ***
## I(cos(perioda2 * t))  0.2267701  0.0546668   4.148 4.58e-05 ***
## I(cos(perioda3 * t)) -0.1603372  0.0546668  -2.933  0.00366 ** 
## I(cos(perioda4 * t)) -0.1244814  0.0546668  -2.277  0.02362 *  
## I(sin(perioda5 * t))  0.1724847  0.0546669   3.155  0.00180 ** 
## I(cos(perioda6 * t))  0.1297305  0.0546668   2.373  0.01839 *  
## ---
## Signif. codes:  0 '***' 0.001 '**' 0.01 '*' 0.05 '.' 0.1 ' ' 1
## 
## Residual standard error: 0.6233 on 253 degrees of freedom
## Multiple R-squared:  0.3405, Adjusted R-squared:  0.3248 
## F-statistic: 21.77 on 6 and 253 DF,  p-value: < 2.2e-16
\end{verbatim}

\includegraphics{porocilo_seminar_D_files/figure-latex/unnamed-chunk-15-2.pdf}
\includegraphics{porocilo_seminar_D_files/figure-latex/unnamed-chunk-15-3.pdf}

\includegraphics{porocilo_seminar_D_files/figure-latex/unnamed-chunk-16-1.pdf}
\includegraphics{porocilo_seminar_D_files/figure-latex/unnamed-chunk-16-2.pdf}

\begin{verbatim}
## Daniell(2) 
## coef[-2] = 0.2
## coef[-1] = 0.2
## coef[ 0] = 0.2
## coef[ 1] = 0.2
## coef[ 2] = 0.2
\end{verbatim}

\begin{verbatim}
## mDaniell(1) 
## coef[-1] = 0.25
## coef[ 0] = 0.50
## coef[ 1] = 0.25
\end{verbatim}

\begin{verbatim}
## Daniell(1,1) 
## coef[-2] = 0.1111
## coef[-1] = 0.2222
## coef[ 0] = 0.3333
## coef[ 1] = 0.2222
## coef[ 2] = 0.1111
\end{verbatim}

\begin{verbatim}
## mDaniell(1,1) 
## coef[-2] = 0.0625
## coef[-1] = 0.2500
## coef[ 0] = 0.3750
## coef[ 1] = 0.2500
## coef[ 2] = 0.0625
\end{verbatim}

\includegraphics{porocilo_seminar_D_files/figure-latex/unnamed-chunk-16-3.pdf}
\includegraphics{porocilo_seminar_D_files/figure-latex/unnamed-chunk-16-4.pdf}
\includegraphics{porocilo_seminar_D_files/figure-latex/unnamed-chunk-16-5.pdf}

\begin{verbatim}
##   [1]   1   2   3   4   5   6   7   8   9  10  11 102 101 100 103 114  99  67
##  [19] 113  98  66  12 115  68 104 112  65  13  97  64  69 116 111 105  14  24
##  [37]  70  63 117  96  23 106 110  25  22  71 109  72 108  15 107 118  21  95
##  [55]  26  20  62  73  94  16 119  19  93  75  74  76  92  27 130 131  90  50
##  [73]  18  89  17 129  77 128  91  51 132  52  88  53 133 120  48  49  61 134
##  [91] 127  87  78  47 135 126 125  28  86 124  80  79  85 123 122  54  46  81
## [109] 121  45  84  42  44  41  39  82  40  43  29  83  55  38  60  59  37  58
## [127]  56  36  30  57  35  34  31  33  32
\end{verbatim}

\begin{verbatim}
## [1] NA NA NA NA
\end{verbatim}

\begin{verbatim}
## [1] NA NA NA NA
\end{verbatim}

\includegraphics{porocilo_seminar_D_files/figure-latex/unnamed-chunk-17-1.pdf}

\begin{enumerate}
\def\labelenumi{\arabic{enumi}.}
\setcounter{enumi}{2}
\tightlist
\item
  \emph{Narišite graf rezidualov in komentirajte, ali so videti
  stacionarni. Stacionarnost tudi preizkusite z uporabo ustreznih
  statističnih metod.}
\end{enumerate}

\begin{verbatim}
## [1] 0.324818
\end{verbatim}

\includegraphics{porocilo_seminar_D_files/figure-latex/unnamed-chunk-18-1.pdf}

\begin{verbatim}
## 
##  Augmented Dickey-Fuller Test
## 
## data:  d.res
## Dickey-Fuller = -5.9954, Lag order = 6, p-value = 0.01
## alternative hypothesis: stationary
\end{verbatim}

\begin{verbatim}
## [1] "Augmented Dickey-Fuller Test ne zavrne stacionarnosti."
\end{verbatim}

\begin{enumerate}
\def\labelenumi{\arabic{enumi}.}
\setcounter{enumi}{3}
\tightlist
\item
  \emph{Na rezidualih naredite grafikona ACF in PACF in na njuni podlagi
  predlagajte vsaj en model vrste AR(p) ali MA(q).}
\end{enumerate}

\includegraphics{porocilo_seminar_D_files/figure-latex/unnamed-chunk-19-1.pdf}
\includegraphics{porocilo_seminar_D_files/figure-latex/unnamed-chunk-19-2.pdf}

\begin{verbatim}
## [1] "Videti je, da nimamo avtokorelacije."
\end{verbatim}

\begin{enumerate}
\def\labelenumi{\arabic{enumi}.}
\setcounter{enumi}{4}
\tightlist
\item
  \emph{Na podlagi Yule--Walkerjevih cenilk in kriterija AIC izberite
  najboljši model AR(p). Primerjajte ga z najboljšim modelom ARMA(p, q)
  za p + q ≤ 3 po kriteriju AIC (pozor: kriterij AIC je lahko definiran
  drugače od postopka do postopka). Če je videti smiselno, pa namesto
  tega uporabite model GARCH.}
\end{enumerate}

\begin{verbatim}
## [1] "Profesor Toman je bil bolj eleganten in se je iskanja rešitve lotil s preprosto zanko."
\end{verbatim}

\begin{verbatim}
## 
## Call:
## arima(x = d.res, order = c(0, 0, 0))
## 
## Coefficients:
##       intercept
##          0.0000
## s.e.     0.0381
## 
## sigma^2 estimated as 0.378:  log likelihood = -242.45,  aic = 488.91
\end{verbatim}

\begin{verbatim}
## [1] "Algoritem vrne predlog za model ARMA(0,0)."
\end{verbatim}

\begin{enumerate}
\def\labelenumi{\arabic{enumi}.}
\setcounter{enumi}{5}
\tightlist
\item
  \emph{Izberite »optimalni« model in ocenite vse njegove parametre.}
\end{enumerate}

\begin{verbatim}
## [1] "OPTIMALNI MODEL: izberemo tisti model, ki ima najnižji aic. V najinem primeru je to ARMA(0,0)."
\end{verbatim}

\includegraphics{porocilo_seminar_D_files/figure-latex/unnamed-chunk-21-1.pdf}
\includegraphics{porocilo_seminar_D_files/figure-latex/unnamed-chunk-21-2.pdf}

\begin{verbatim}
## 
##  Shapiro-Wilk normality test
## 
## data:  best$residuals
## W = 0.71579, p-value < 2.2e-16
\end{verbatim}

\includegraphics{porocilo_seminar_D_files/figure-latex/unnamed-chunk-21-3.pdf}

\begin{verbatim}
## [1] "Shapirov test zavrne hipotezo, torej ne gre za normalno porazdelitev, kar je očitno tudi z grafa."
\end{verbatim}

\begin{enumerate}
\def\labelenumi{\arabic{enumi}.}
\setcounter{enumi}{6}
\tightlist
\item
  \emph{Oglejte si ostanke po vašem modelu in komentirajte, ali so
  videti kot beli šum. Primerjajte njihovo porazdelitev z normalno.}
  \includegraphics{porocilo_seminar_D_files/figure-latex/unnamed-chunk-22-1.pdf}
\end{enumerate}

\begin{verbatim}
## 
##  Box-Pierce test
## 
## data:  d.res
## X-squared = 0.46321, df = 1, p-value = 0.4961
\end{verbatim}

\begin{verbatim}
## 
##  Box-Ljung test
## 
## data:  d.res
## X-squared = 0.46858, df = 1, p-value = 0.4936
\end{verbatim}

\begin{verbatim}
## [1] "Imamo white noise!"
\end{verbatim}

\begin{enumerate}
\def\labelenumi{\arabic{enumi}.}
\setcounter{enumi}{7}
\tightlist
\item
  \emph{Z uporabo izbranega modela in pod predpostavko normalnosti z
  R-ovo funkcijo predict konstruirajte 90\% napovedni interval za
  naslednjo vrednost. Ne pozabite vračunati tudi odstranjenega trenda in
  sezonskosti.}
\end{enumerate}

\includegraphics{porocilo_seminar_D_files/figure-latex/unnamed-chunk-23-1.pdf}
\includegraphics{porocilo_seminar_D_files/figure-latex/unnamed-chunk-23-2.pdf}

\begin{enumerate}
\def\labelenumi{\arabic{enumi}.}
\setcounter{enumi}{8}
\tightlist
\item
  \emph{Dobljeni napovedni interval primerjajte z napovednim intervalom,
  ki bi ga dobili, če bi naivno privzeli, da so podatki kar Gaussov beli
  šum -- pred in po odstranitvi trenda in sezonskosti.}
\end{enumerate}

\includegraphics{porocilo_seminar_D_files/figure-latex/unnamed-chunk-24-1.pdf}
\includegraphics{porocilo_seminar_D_files/figure-latex/unnamed-chunk-24-2.pdf}
\includegraphics{porocilo_seminar_D_files/figure-latex/unnamed-chunk-24-3.pdf}
\includegraphics{porocilo_seminar_D_files/figure-latex/unnamed-chunk-24-4.pdf}

\end{document}
